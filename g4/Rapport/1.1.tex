%%%%%%%%%%%%%%%%%%%%%%%%%%%%%%%%%%%%%%%%%%%%%%%%%%
%%%%%%%%%%%%%%%%%%%%%%%%%%%%%%%%%%%%%%%%%%%%%%%%%%
%Implement handlers for TLB related exceptions in Buenos. 
%%%%%%%%%%%%%%%%%%%%%%%%%%%%%%%%%%%%%%%%%%%%%%%%%%
%%%%%%%%%%%%%%%%%%%%%%%%%%%%%%%%%%%%%%%%%%%%%%%%%%
Da opgaveteksten siger, load og store håndteres på samme måde, har vi valgt at lave en hjælpefunktion, 
der håndterer begge tilfælde. Funktionen hedder \verb+tlb_exception_handler+ og starter med at 
sammenligne den virtuelle adresse som fejlen er meldt på, med adresserne i pagetables i den angivne 
tråd. Hvis vi finder adressen, sættes den ind i TLB, ellers behandler vi den som en access violation, 
henholdsvis i bruger-tråds tilstand eller kerne tilstand. Da vi behandler både load og store i handleren, 
findes der fejlbeskeder for begge her. \\

\noindent \verb+tlb_load_exception+ og \verb+tlb_store_exception+ kalder vores handler, og bruger 
derfor den samme kode. \verb+tlb_modified_exception+ håndteres på næsten samme måde, som i vores 
handler, hvor vi også sørger for at håndtere access violation i bruger-tråds tilstande samt kernel panic i 
kerne tilstande. Derfor er det samme kode, der blot er kopieret op, dog uden genbrug af de if-sætninger, 
der håndterede de 2 forskellige fejlbeskeder for henholdsvis store og load. 