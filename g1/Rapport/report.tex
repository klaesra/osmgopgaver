% \begin{sloppypar} <-- fikser de afsnit, hvor der bruges \lstinline og
% der spilder ud i marginen.


%\includeonly{userguide}

\documentclass[a4paper,10pt]{article}

% NOTE: Missing packages? Try texlive-latex-extra
% TIP (vim): map <F5> :!make view<RETURN><RETURN> 

%\usepackage[danish]{babel,varioref}
\usepackage[T1]{fontenc}
\usepackage[utf8]{inputenc}

% fonts
%\usepackage{charter}
%\usepackage{euler}
%\usepackage{palatino}
\usepackage{mathpazo}

\usepackage{float} % for [H] figure option'en
\usepackage{graphicx}
\usepackage{longtable}
\usepackage[breaklinks=true,colorlinks=false,urlcolor=blue]{hyperref}
\usepackage{breakurl} % due to http://www.tex.ac.uk/cgi-bin/texfaq2html?label=breaklinks
\usepackage{color}
\usepackage{amsmath}

\usepackage[numbers]{natbib}
\usepackage{textcomp}
\usepackage{listings}
\usepackage{verbatim}
\usepackage[table]{xcolor}
\usepackage[textfont={small,it}]{caption}
\usepackage{subfig}
\usepackage[]{algorithm2e}



\definecolor{listinggray}{gray}{0.9}
\definecolor{lbcolor}{rgb}{0.9,0.9,0.9}

\definecolor{darkgreen}{rgb}{0,0.5,0}
\definecolor{darkred}{rgb}{255,1,0.120}
\lstset{language=Pascal,
        %captionpos=b,
	tabsize=4,
	frame=lines,
	%keywordstyle=\color{blue},
	commentstyle=\color{darkgreen},
	stringstyle=\color{red},
	numberstyle=\tiny,
	%numbersep=5pt,
	breaklines=true,
	showstringspaces=false,
	basicstyle=\ttfamily,
	%title= File: \lstname,
	emph={label},
}

\title{Styresystemer og Multiprogrammering}
\author{Jenny-Margrethe Vej, Klaes Bo Rasmussen}
\makeatletter % enable use of \@ special variables
\hypersetup{
	pdftitle={\@title},
	pdfauthor={\@author}
}

\newcommand{\code}[1] {\lstinline!#1!}

\newenvironment{codeblock}[1][0.99] {
	\begin{center}
	\begin{minipage}
	{#1\textwidth}
}{
	\end{minipage}
	\end{center}
}

\begin{document}
\thispagestyle{empty}
\vspace*{\stretch{0}}
\begin{flushright}
   {\Huge\textbf{\@title}}\\[3mm]
   \rule{\linewidth}{2mm}\\[3mm]
   {\Large\textbf{\textit{G-opgave \#1}}}\\
   \vspace{12cm}
   {\normalsize \textbf{Jenny-Margrethe Vej}
   \\(\url{rwj935@alumni.ku.dk})}
   \\
   {\normalsize \textbf{Klaes Bo Rasmussen}
   \\(\url{twb822@alumni.ku.dk})}
   \\
	\vspace*{2cm}
   {\normalsize Datalogisk Institut, Københavns Universitet}\\
   {\normalsize Blok 3 - 2013}\\
\end{flushright}
%\vspace*{\stretch{2}}
\clearpage

\section*{A simple queue using a linked list}
Vi har bygget koden op i samme rækkefølge som \verb+queue.h+, og har ydermere lavet en 
\verb+Makefile+, således man kan køre vores c-fil ved brug af kommandoen \verb+make+ i sin terminal.
\subsection*{(a)}
%%%%%%%%%%%%%%%%%%%%%%%%%%%%%%%%%%%%%%%%%%%%%%%%%%
%%%%%%%%%%%%%%%%%%%%%%%%%%%%%%%%%%%%%%%%%%%%%%%%%%
%Implement functions enqueue and dequeue (described above), and a length 
%function.
%%%%%%%%%%%%%%%%%%%%%%%%%%%%%%%%%%%%%%%%%%%%%%%%%%
%%%%%%%%%%%%%%%%%%%%%%%%%%%%%%%%%%%%%%%%%%%%%%%%%%
Vi har bygget koden op i samme rækkefølge som \verb+queue.h+ 

\subsection*{(b)}
%%%%%%%%%%%%%%%%%%%%%%%%%%%%%%%%%%%%%%%%%%%%%%%%%%
%%%%%%%%%%%%%%%%%%%%%%%%%%%%%%%%%%%%%%%%%%%%%%%%%%
% Explain why enqueue and dequeue must use an argument of type QNode**
%%%%%%%%%%%%%%%%%%%%%%%%%%%%%%%%%%%%%%%%%%%%%%%%%%
%%%%%%%%%%%%%%%%%%%%%%%%%%%%%%%%%%%%%%%%%%%%%%%%%%
\subsection*{(c)}
%%%%%%%%%%%%%%%%%%%%%%%%%%%%%%%%%%%%%%%%%%%%%%%%%%
%%%%%%%%%%%%%%%%%%%%%%%%%%%%%%%%%%%%%%%%%%%%%%%%%%
% Implement the sum function, which returns a sum of values of data in the queue, 
% and uses function pointer ∗val to get the values.
% For example, if the value function maps a to 1, b to 2, and c to 3, 
% and the queue qu contains [a,b,c] (see picture above), 
% the call sum(qu,value) will return 6. 
% Called with a constant function which always returns 1, 
% the sum function will return the queue’s length.
%%%%%%%%%%%%%%%%%%%%%%%%%%%%%%%%%%%%%%%%%%%%%%%%%%
%%%%%%%%%%%%%%%%%%%%%%%%%%%%%%%%%%%%%%%%%%%%%%%%%%
\section*{Buenos system calls for basic I/O}
\subsection*{(a)}
%%%%%%%%%%%%%%%%%%%%%%%%%%%%%%%%%%%%%%%%%%%%%%%%%%
%%%%%%%%%%%%%%%%%%%%%%%%%%%%%%%%%%%%%%%%%%%%%%%%%%
% Implement system calls read and write with the behaviour outlined below 
% (also described in Section 6.4 of the Buenos roadmap). 
% The system call interface in tests/lib.c already provides wrappers for these calls, 
% and their system call num- bers are defined in proc/syscall.h. 
% It is not necessary to make the system call code ”bullet-proof ” 
% (as it is called in the Buenos roadmap).
%
%% i. int syscall_read(int fhandle, void *buffer, int length);
% Read at most length bytes from the file identified by fhandle 
% (at the current file position) into buffer,  advancing the file position. 
% Returns the number of bytes actually read (before reaching the end of the file), 
% or a negative value on error. 
% Simplification: Your implementation should only read from FILEHANDLE STDIN, 
% (number 0 in proc/syscall.h), using the generic character device driver.
%
%% ii. int syscall_write(int fhandle, const void *buffer, int length); 
% Write length bytes from buffer to the open file identified by fhandle, 
% starting at the current position and advancing the position. 
% Returns the number of bytes actually written, or a negative value on error.
% Simplification: Your implementation should only write to FILEHANDLE STDOUT, 
% (number 1 in proc/syscall.h), using the generic character device driver.
% Look at the code inside init startup fallback (file init/main.c) to see how to acquire and use the generic 
% character device (also see drivers/gcd.h).
%%%%%%%%%%%%%%%%%%%%%%%%%%%%%%%%%%%%%%%%%%%%%%%%%%
%%%%%%%%%%%%%%%%%%%%%%%%%%%%%%%%%%%%%%%%%%%%%%%%%%

For at besvare denne delopgave, har vi tilføjet de ønskede funktioner til filen \verb+proc/syscall.c+. \\

\noindent Både \verb+read+ og \verb+write+ er implementeret ud fra simplificeringen i opgaveteksten. Vi 
gør ikke brug af \verb+fhandle+ i denne opgave, da vi kun bruger standardfilerne som via \verb+gcd+ 
håndterer handles automatisk. For at undgå advarsler, når vi kører filerne, starter vi med bare at skrive 
\verb+fhandle = fhandle;+ både til \verb+syscall_read+ og \verb+syscall_write+.\\

\noindent Vi har fulgt rådet i opgaveteksten, og brugt samme metoder til håndtering af \verb+gcd+, hvorfor 
man også vil se, at flere af linjerne indeholder faktisk det samme kode. \\

\noindent Udover de 2 funktioner ovenfor, har vi også tilføjet to cases; \verb+SYSCALL_READ+ og 
\verb+SYSCALL_WRITE+. Her sørger vi for at fange de rigtige cases, og samtidig sørger vi for at putte 
returværdien i register \verb+v0+, jf. section 6.4 i Buenos Roadmap. Fra vores cases bliver vores 2 
funktioner \verb+syscall_read+ og \verb+syscall_write+ kaldt med de korrekte argumenter fra registrene
\verb+a1--a3+.
\subsection*{(b)}
%%%%%%%%%%%%%%%%%%%%%%%%%%%%%%%%%%%%%%%%%%%%%%%%%%
%%%%%%%%%%%%%%%%%%%%%%%%%%%%%%%%%%%%%%%%%%%%%%%%%%
% Test the implemented system calls by a small C program readwrite.c in directory tests/ 
% (see halt.c there for an example). 
% Copy the compiled program to the Buenos disk using tfstool and start it using boot argument 
% initprog=[root]readwrite (see page 6ff of the Buenos roadmap for instructions and explanations).
%%%%%%%%%%%%%%%%%%%%%%%%%%%%%%%%%%%%%%%%%%%%%%%%%%
%%%%%%%%%%%%%%%%%%%%%%%%%%%%%%%%%%%%%%%%%%%%%%%%%%
Vores testfil ligger i mappen tests.\\

\noindent Når vi giver vores \verb+syscall_write+ et antal karakterer at skrive til \verb+stdout+, finder den
de første ledige, uden at vente på eventuel input undervejs. Det betyder, at hvis vi skrev 60, ville den 
finde 60 tegn, udskrive dem og først derefter afvikle næste systemkald. Man kan altså sørge for, at den 
kun skriver det, man vil have, ved at give den korrekte længde med på den streng, man vil udskrive.\\

\noindent \verb+syscall_read+ læser en karakter fra \verb+stdin+ og gemmer derefter i en specificeret 
buffer. \\

\noindent Vores udførte tests giver de forventede output, og vi konkluderer derfor, vi har løst opgaven som 
ønsket. 

%\pagenumbering{roman}
%\setcounter{tocdepth}{3}
\thispagestyle{empty}
\clearpage
\renewcommand{\labelitemi}{$\cdot$}
\begin{thebibliography}{9}
\end{thebibliography}


%\setlength{\parindent}{2em}
%\setlength{\parskip}{2ex}
%\renewcommand{\arraystretch}{1.25}
%\setcounter{page}{1}
\end{document}

