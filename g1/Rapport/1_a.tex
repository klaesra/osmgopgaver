%%%%%%%%%%%%%%%%%%%%%%%%%%%%%%%%%%%%%%%%%%%%%%%%%%
%%%%%%%%%%%%%%%%%%%%%%%%%%%%%%%%%%%%%%%%%%%%%%%%%%
%Implement functions enqueue and dequeue (described above), and a length 
%function.
%%%%%%%%%%%%%%%%%%%%%%%%%%%%%%%%%%%%%%%%%%%%%%%%%%
%%%%%%%%%%%%%%%%%%%%%%%%%%%%%%%%%%%%%%%%%%%%%%%%%%
Vores længde-funktion returnerer antallet af elementer i en kø. Den starter med længden 0, og er der 1 
eller flere elementer, laves der en pointer til den nuværende node i køen, som vi kalder current, og lægger 
en til. Herefter går vi ned i while-løkken, som bliver ved at tjekke op på, om elementerne i køen er ens, 
lægger en til længden hver gang, de ikke er ens, og stopper så, når den rammer første element i køen 
igen. \\

\noindent \verb+enqueue+ tilføjer et element til enden af vores kø ved først at allokere pladsen til det vi 
gerne vil indsætte. Hvis det ikke er muligt at allokere den ønskede hukommelse, printer vi beskeden "Out 
of memory". I næste if-sætning, kigger vi på, om køen er tom, er den det, indsætter vi et element, som så
vil pege på sig selv. Hvis der i forvejen er flere elementer, indsætter vi stadig det ønskede element, men 
opdaterer samtidig alle pegerne, så de står korrekt efter tilføjelsen af det nye element. \\

\noindent \verb+dequeue+ gør brug af samme tankegang som ovenfor, bare "med modsat fortegn" - vi 
fjerner altså et element her i stedet for at tilføje det, og sørger for også at frigøre pladsen fra det element, 
vi har fjernet.\\

\noindent Vi har testet alle 3 funktioner i \verb+main()+, hvor vi først indsætter elementer i køen, for derefter
at tjekke længden. Derefter fjerner vi elementer, og tjekker samtidig, at længde-funktionen også virker her.