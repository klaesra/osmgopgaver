%%%%%%%%%%%%%%%%%%%%%%%%%%%%%%%%%%%%%%%%%%%%%%%%%%
%%%%%%%%%%%%%%%%%%%%%%%%%%%%%%%%%%%%%%%%%%%%%%%%%%
% Implement the sum function, which returns a sum of values of data in the queue, 
% and uses function pointer ∗val to get the values.
% For example, if the value function maps a to 1, b to 2, and c to 3, 
% and the queue qu contains [a,b,c] (see picture above), 
% the call sum(qu,value) will return 6. 
% Called with a constant function which always returns 1, 
% the sum function will return the queue’s length.
%%%%%%%%%%%%%%%%%%%%%%%%%%%%%%%%%%%%%%%%%%%%%%%%%%
%%%%%%%%%%%%%%%%%%%%%%%%%%%%%%%%%%%%%%%%%%%%%%%%%%

Vores sum-funktion er bygget op på samme måde som \verb+length+, med samme måde at løbe gennem 
køen på. Forskellen består selvfølgelig her i, at sum-funktionen summerer værdien af dataen fra køen. 
Funktionen er testet sammen med de andre funktioner, da vi alligevel skulle have nogle elementer i køen 
at teste ud fra. \\

\noindent \verb+sum+ tager 2 argumenter, og det ene satte vi til at være den indbyggede funktion 
\verb+strlen+. Dog måtte vi lige lave en hjælpefunktion \verb+mystrlen+, da datatypen eksplicit skal være
 en \verb+char+, og ikke \verb+Data+. 