%%%%%%%%%%%%%%%%%%%%%%%%%%%%%%%%%%%%%%%%%%%%%%%%%%
%%%%%%%%%%%%%%%%%%%%%%%%%%%%%%%%%%%%%%%%%%%%%%%%%%
% Test the implemented system calls by a small C program readwrite.c in directory tests/ 
% (see halt.c there for an example). 
% Copy the compiled program to the Buenos disk using tfstool and start it using boot argument 
% initprog=[root]readwrite (see page 6ff of the Buenos roadmap for instructions and explanations).
%%%%%%%%%%%%%%%%%%%%%%%%%%%%%%%%%%%%%%%%%%%%%%%%%%
%%%%%%%%%%%%%%%%%%%%%%%%%%%%%%%%%%%%%%%%%%%%%%%%%%
Vores testfil ligger i mappen tests.\\

\noindent Når vi giver vores \verb+syscall_write+ et antal karakterer at skrive til \verb+stdout+, finder den
de første ledige, uden at vente på eventuel input undervejs. Det betyder, at hvis vi skrev 60, ville den 
finde 60 tegn, udskrive dem og først derefter afvikle næste systemkald. Man kan altså sørge for, at den 
kun skriver det, man vil have, ved at give den korrekte længde med på den streng, man vil udskrive.\\

\noindent \verb+syscall_read+ læser en karakter fra \verb+stdin+ og gemmer derefter i en specificeret 
buffer. \\

\noindent Vores udførte tests giver de forventede output, og vi konkluderer derfor, vi har løst opgaven som 
ønsket. 