%%%%%%%%%%%%%%%%%%%%%%%%%%%%%%%%%%%%%%%%%%%%%%%%%%
%%%%%%%%%%%%%%%%%%%%%%%%%%%%%%%%%%%%%%%%%%%%%%%%%%
% Implement system calls read and write with the behaviour outlined below 
% (also described in Section 6.4 of the Buenos roadmap). 
% The system call interface in tests/lib.c already provides wrappers for these calls, 
% and their system call num- bers are defined in proc/syscall.h. 
% It is not necessary to make the system call code ”bullet-proof ” 
% (as it is called in the Buenos roadmap).
%
%% i. int syscall_read(int fhandle, void *buffer, int length);
% Read at most length bytes from the file identified by fhandle 
% (at the current file position) into buffer,  advancing the file position. 
% Returns the number of bytes actually read (before reaching the end of the file), 
% or a negative value on error. 
% Simplification: Your implementation should only read from FILEHANDLE STDIN, 
% (number 0 in proc/syscall.h), using the generic character device driver.
%
%% ii. int syscall_write(int fhandle, const void *buffer, int length); 
% Write length bytes from buffer to the open file identified by fhandle, 
% starting at the current position and advancing the position. 
% Returns the number of bytes actually written, or a negative value on error.
% Simplification: Your implementation should only write to FILEHANDLE STDOUT, 
% (number 1 in proc/syscall.h), using the generic character device driver.
% Look at the code inside init startup fallback (file init/main.c) to see how to acquire and use the generic 
% character device (also see drivers/gcd.h).
%%%%%%%%%%%%%%%%%%%%%%%%%%%%%%%%%%%%%%%%%%%%%%%%%%
%%%%%%%%%%%%%%%%%%%%%%%%%%%%%%%%%%%%%%%%%%%%%%%%%%