% \begin{sloppypar} <-- fikser de afsnit, hvor der bruges \lstinline og
% der spilder ud i marginen.


%\includeonly{userguide}

\documentclass[a4paper,10pt]{article}

% NOTE: Missing packages? Try texlive-latex-extra
% TIP (vim): map <F5> :!make view<RETURN><RETURN> 

%\usepackage[danish]{babel,varioref}
\usepackage[T1]{fontenc}
\usepackage[utf8]{inputenc}

% fonts
%\usepackage{charter}
%\usepackage{euler}
%\usepackage{palatino}
\usepackage{mathpazo}

\usepackage{float} % for [H] figure option'en
\usepackage{graphicx}
\usepackage{longtable}
\usepackage[breaklinks=true,colorlinks=false,urlcolor=blue]{hyperref}
\usepackage{breakurl} % due to http://www.tex.ac.uk/cgi-bin/texfaq2html?label=breaklinks
\usepackage{color}
\usepackage{amsmath}

\usepackage[numbers]{natbib}
\usepackage{textcomp}
\usepackage{listings}
\usepackage{verbatim}
\usepackage[table]{xcolor}
\usepackage[textfont={small,it}]{caption}
\usepackage{subfig}
%\usepackage[]{algorithm2e}



\definecolor{listinggray}{gray}{0.9}
\definecolor{lbcolor}{rgb}{0.9,0.9,0.9}

\definecolor{darkgreen}{rgb}{0,0.5,0}
\definecolor{darkred}{rgb}{255,1,0.120}
\lstset{language=Pascal,
        %captionpos=b,
	tabsize=4,
	frame=lines,
	%keywordstyle=\color{blue},
	commentstyle=\color{darkgreen},
	stringstyle=\color{red},
	numberstyle=\tiny,
	%numbersep=5pt,
	breaklines=true,
	showstringspaces=false,
	basicstyle=\ttfamily,
	%title= File: \lstname,
	emph={label},
}

\title{Styresystemer og Multiprogrammering}
\author{Jenny-Margrethe Vej, Klaes Bo Rasmussen}
\makeatletter % enable use of \@ special variables
\hypersetup{
	pdftitle={\@title},
	pdfauthor={\@author}
}

\newcommand{\code}[1] {\lstinline!#1!}

\newenvironment{codeblock}[1][0.99] {
	\begin{center}
	\begin{minipage}
	{#1\textwidth}
}{
	\end{minipage}
	\end{center}
}

\begin{document}
\thispagestyle{empty}
\vspace*{\stretch{0}}
\begin{flushright}
   {\Huge\textbf{\@title}}\\[3mm]
   \rule{\linewidth}{2mm}\\[3mm]
   {\Large\textbf{\textit{G-opgave \#2}}}\\
   \vspace{12cm}
   {\normalsize \textbf{Jenny-Margrethe Vej}
   \\(\url{rwj935@alumni.ku.dk})}
   \\
   {\normalsize \textbf{Klaes Bo Rasmussen}
   \\(\url{twb822@alumni.ku.dk})}
   \\
	\vspace*{2cm}
   {\normalsize Datalogisk Institut, Københavns Universitet}\\
   {\normalsize Blok 3 - 2013}\\
\end{flushright}
%\vspace*{\stretch{2}}
\clearpage

\section*{Types and Functions for User Processes in Buenos}

1. We defined the required datastructure
\verb+process\_control\_block\_t+ within process.h. For now it only contains
what we belive to be the required fields to complete the given
task. 

We can save it state, given by the \verb+enum process\_state\_t+. 

It keeps a return value, which is
used in \verb+process\_join+, \verb+process\_finish and it
also acts as the exit code for \verb+syscall\_exit+.

It contains a small buffer which is used for saving the name of the
executeable to be run, currently programs can have a name of maximum
32 chars.

Last, and most importantly, it keeps track of which process it
is. This process id is used any time a process is called upon for any
functionality. 

\vspace{1pc}

2. We have implemented all the given functions in the second task
inside the file \verb+process.c+. We will not say much about this,
mostly explain the problem we came to when we was able to test the
code. 

So, first, everything does compile. But, I have added the ``hw'' program
and the ``exec'' programs from the ``tests'' directory, to the disk,
and running exec presents a quite big problem. Looking at the exec.c
file, we figured that the program would run the ``hw'' program by
default. But when exec is run, the text that is thrown with it, which
is supposed to be the name of the program to be run, is instead
``[arkimedes]exec'' itself. We were unable to fix this, and thus
cannot test the rest of the functionality, suspected problem is the
way we edited \verb+start\_process+ to account for the new data
structure. But this will be looked in to and fixed asap!

\vspace{1pc}

\section*{System Calls for User Process Control in Buenos}

There is not much in this task, but maybe something have gone wrong
here too since obviously running executables does not work. Otherwise,
the system calls merely provide a shell for calling the functions
defined in the first part of the task. All are defined in
\verb+syscall.c+, where the functions are defined first, and then
the cases in the actual syscall.


%\pagenumbering{roman}
%\setcounter{tocdepth}{3}
%\thispagestyle{empty}
%\clearpage
%\renewcommand{\labelitemi}{$\cdot$}
%\input{bibliography}

%\setlength{\parindent}{2em}
%\setlength{\parskip}{2ex}
%\renewcommand{\arraystretch}{1.25}
%\setcounter{page}{1}
\end{document}

