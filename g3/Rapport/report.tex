% \begin{sloppypar} <-- fikser de afsnit, hvor der bruges \lstinline og
% der spilder ud i marginen.


%\includeonly{userguide}

\documentclass[a4paper,10pt]{article}

% NOTE: Missing packages? Try texlive-latex-extra
% TIP (vim): map <F5> :!make view<RETURN><RETURN> 

%\usepackage[danish]{babel,varioref}
\usepackage[T1]{fontenc}
\usepackage[utf8]{inputenc}

% fonts
%\usepackage{charter}
%\usepackage{euler}
%\usepackage{palatino}
\usepackage{mathpazo}

\usepackage{float} % for [H] figure option'en
\usepackage{graphicx}
\usepackage{longtable}
\usepackage[breaklinks=true,colorlinks=false,urlcolor=blue]{hyperref}
\usepackage{breakurl} % due to http://www.tex.ac.uk/cgi-bin/texfaq2html?label=breaklinks
\usepackage{color}
\usepackage{amsmath}

\usepackage[numbers]{natbib}
\usepackage{textcomp}
\usepackage{listings}
\usepackage{verbatim}
\usepackage[table]{xcolor}
\usepackage[textfont={small,it}]{caption}
\usepackage{subfig}
%\usepackage[]{algorithm2e}



\definecolor{listinggray}{gray}{0.9}
\definecolor{lbcolor}{rgb}{0.9,0.9,0.9}

\definecolor{darkgreen}{rgb}{0,0.5,0}
\definecolor{darkred}{rgb}{255,1,0.120}
\lstset{language=Pascal,
        %captionpos=b,
	tabsize=4,
	frame=lines,
	%keywordstyle=\color{blue},
	commentstyle=\color{darkgreen},
	stringstyle=\color{red},
	numberstyle=\tiny,
	%numbersep=5pt,
	breaklines=true,
	showstringspaces=false,
	basicstyle=\ttfamily,
	%title= File: \lstname,
	emph={label},
}

\title{Styresystemer og Multiprogrammering}
\author{Jenny-Margrethe Vej, Klaes Bo Rasmussen}
\makeatletter % enable use of \@ special variables
\hypersetup{
	pdftitle={\@title},
	pdfauthor={\@author}
}

\newcommand{\code}[1] {\lstinline!#1!}

\newenvironment{codeblock}[1][0.99] {
	\begin{center}
	\begin{minipage}
	{#1\textwidth}
}{
	\end{minipage}
	\end{center}
}

\begin{document}
\thispagestyle{empty}
\vspace*{\stretch{0}}
\begin{flushright}
   {\Huge\textbf{\@title}}\\[3mm]
   \rule{\linewidth}{2mm}\\[3mm]
   {\Large\textbf{\textit{G-opgave \#3}}}\\
   \vspace{12cm}
   {\normalsize \textbf{Jenny-Margrethe Vej}
   \\(\url{rwj935@alumni.ku.dk})}
   \\
   {\normalsize \textbf{Klaes Bo Rasmussen}
   \\(\url{twb822@alumni.ku.dk})}
   \\
	\vspace*{2cm}
   {\normalsize Datalogisk Institut, Københavns Universitet}\\
   {\normalsize Blok 3 - 2013}\\
\end{flushright}
%\vspace*{\stretch{2}}
\clearpage


\section{Locks and condition variables for kernel threads in Buenos}
1. First task should be completed, it goes as follows:

\vspace{1pc}

\verb+lock_reset+ 

This resets the spinlock and the \verb+locked+ variable in the given
\verb+lock+.

\vspace{0.5pc}

\verb+lock_acquire+

Saves the interrupt status, setting it to disabled. Then acquires
spinlock for the given lock. It runs and keeps checking if the lock is
still locked. If so, it adds the lock to the sleep queue, releases the
spinlock and switches thread. When it is no longer locked, it locks it
again so its locked to itself and the code can run without
disruptions. 

\vspace{0.5pc}

\verb+lock_release+

Disables interrupts, saving the status. Then acquires the spinlock. If
lock is no longer locked, wake it up and continue to release spinlock
and set the interrupt status back to its former glory.

\vspace{1pc}

2. We don't know how to test this, but at least we get what this was
supposed to do now.

\vspace{0.5pc}

\verb+condition_init(cond_t *cond)+

This doesn't need to do anything.

\vspace{0.5pc}

\verb+condition_wait(cond_t *cond, lock_t *lock)+

Since it is required that this is only called when the lock is already
acquired, we need to release it first. Then add the condition to the
sleepq. Switch to a new thread until we return back here, and then
reacquire the lock so it can be released when we get out of the
\verb+condition_wait+.

\vspace{0.5pc}

\verb+condition_signal(cond_t *cond, lock_t *lock)+

\verb+lock+ is set to itself, to satisfy the compiler. Then we signal,
or rather wake up the sleeping condition.

\vspace{0.5pc}

\verb+condition_broadcast(cond_t *cond, lock_t *lock)+

Same as with the \verb+condition_signal+, it just wakes up all sleepers.


\section{One-Lane Synchronisation}

1. When syncronising a stretch of road like this, it would be
preferrable to account for different times of day when traffic amount
changes. But if we consider a simple solution:

Let the first car that gets to the crossing be put into the queue
first, let it pass. If no other cars get to the crossing while the
first is passing it, reset the queue when the car exits.

If more cars come, open up for new cars going in the same
direction. When a car approaches from the other direction, set a timer
of 30 seconds before cars need to stop from the first direction, let
the rest through before opening up for the other direction, given the
same rules. Again, if there are no cars as the last crossing car
leaves the narrow path, reset the queue.

2. We did not manage to complete this part of the task yet. But we did
make an attempt at it. It was hard to figure out exactly how to code
it, but we will attempt to clarify what we did make.

First in the oneLane.h file, we got definitions for some directions,
north, south and no direction at all. Then we create the mutexes and
the cond\_t's that we need to do the task. Some functions that we
first considered necessary to complete the task are also there. These
are specified in:

\vspace{0.5pc}

\verb+oneLane.c+

We define some counters, for approaching and departing busses in each
direction. Also which direction is currently waiting for the other
side.

\vspace{0.5pc}

\verb+stop_other_direction()+

This says, that if we are waiting for some side at all, attempt to
stop busses from that side, to open up for busses from the direction
we want. It's a conditional pthread which is used, to wait while there
are still cars departing from the narrow path in the other
direction. While it does this, it goes to sleep and waits using
pthreads. When there are no more cars, we can unlock again and do
whatever we need to do when one side has stopped running. Same goes
for the other direction, with roles reversed. So far this doesn't
really do much, but we are supposed to tell the approaching cars
somehow to stop coming in from whatever direction the other direction
is, so the counter of departing busses can go to zero.

The \verb+go_this_way+ function is quite uselss as long as we can
already stop the other direction, it is pretty much a different word
for the same thing. It would be more useful if we had to direct the
traffic in some certain way, that an end user would be able to start
the flow of busses in a direction he saw struggling to keep up. But
for an automated solution this is in fact void of usability.

3. Couldn't finish this part without part two.




%\pagenumbering{roman}
%\setcounter{tocdepth}{3}
%\thispagestyle{empty}
%\clearpage
%\renewcommand{\labelitemi}{$\cdot$}
%\input{bibliography}

%\setlength{\parindent}{2em}
%\setlength{\parskip}{2ex}
%\renewcommand{\arraystretch}{1.25}
%\setcounter{page}{1}
\end{document}

