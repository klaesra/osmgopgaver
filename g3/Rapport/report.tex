% \begin{sloppypar} <-- fikser de afsnit, hvor der bruges \lstinline og
% der spilder ud i marginen.


%\includeonly{userguide}

\documentclass[a4paper,10pt]{article}

% NOTE: Missing packages? Try texlive-latex-extra
% TIP (vim): map <F5> :!make view<RETURN><RETURN> 

%\usepackage[danish]{babel,varioref}
\usepackage[T1]{fontenc}
\usepackage[utf8]{inputenc}

% fonts
%\usepackage{charter}
%\usepackage{euler}
%\usepackage{palatino}
\usepackage{mathpazo}

\usepackage{float} % for [H] figure option'en
\usepackage{graphicx}
\usepackage{longtable}
\usepackage[breaklinks=true,colorlinks=false,urlcolor=blue]{hyperref}
\usepackage{breakurl} % due to http://www.tex.ac.uk/cgi-bin/texfaq2html?label=breaklinks
\usepackage{color}
\usepackage{amsmath}

\usepackage[numbers]{natbib}
\usepackage{textcomp}
\usepackage{listings}
\usepackage{verbatim}
\usepackage[table]{xcolor}
\usepackage[textfont={small,it}]{caption}
\usepackage{subfig}
%\usepackage[]{algorithm2e}



\definecolor{listinggray}{gray}{0.9}
\definecolor{lbcolor}{rgb}{0.9,0.9,0.9}

\definecolor{darkgreen}{rgb}{0,0.5,0}
\definecolor{darkred}{rgb}{255,1,0.120}
\lstset{language=Pascal,
        %captionpos=b,
	tabsize=4,
	frame=lines,
	%keywordstyle=\color{blue},
	commentstyle=\color{darkgreen},
	stringstyle=\color{red},
	numberstyle=\tiny,
	%numbersep=5pt,
	breaklines=true,
	showstringspaces=false,
	basicstyle=\ttfamily,
	%title= File: \lstname,
	emph={label},
}

\title{Styresystemer og Multiprogrammering}
\author{Jenny-Margrethe Vej, Klaes Bo Rasmussen}
\makeatletter % enable use of \@ special variables
\hypersetup{
	pdftitle={\@title},
	pdfauthor={\@author}
}

\newcommand{\code}[1] {\lstinline!#1!}

\newenvironment{codeblock}[1][0.99] {
	\begin{center}
	\begin{minipage}
	{#1\textwidth}
}{
	\end{minipage}
	\end{center}
}

\begin{document}
\thispagestyle{empty}
\vspace*{\stretch{0}}
\begin{flushright}
   {\Huge\textbf{\@title}}\\[3mm]
   \rule{\linewidth}{2mm}\\[3mm]
   {\Large\textbf{\textit{G-opgave \#3}}}\\
   \vspace{12cm}
   {\normalsize \textbf{Jenny-Margrethe Vej}
   \\(\url{rwj935@alumni.ku.dk})}
   \\
   {\normalsize \textbf{Klaes Bo Rasmussen}
   \\(\url{twb822@alumni.ku.dk})}
   \\
	\vspace*{2cm}
   {\normalsize Datalogisk Institut, Københavns Universitet}\\
   {\normalsize Blok 3 - 2013}\\
\end{flushright}
%\vspace*{\stretch{2}}
\clearpage


\section{Locks and condition variables for kernel threads in Buenos}
1. This was our understanding of the first task, it may be wrong, but we
would really like some feedback as to how it's wrong, perhaps just
verbal feedback.

\vspace{1pc}

\verb+lock_reset+ 

This resets the spinlock and the \verb+locked+ variable in the given
\verb+lock+.

\vspace{0.5pc}

\verb+lock_acquire+

Saves the interrupt status, setting it to disabled. Then acquires
spinlock for the given lock. It runs and keeps checking if the lock is
still locked. If so, it adds the lock to the sleep queue, releases the
spinlock and switches thread. When it is no longer locked, it locks it
again so its locked to itself and the code can run without
disruptions.

\vspace{0.5pc}

\verb+lock_release+

Disables interrupts, saving the status. Then acquires the spinlock. If
lock is no longer locked, wake it up and continue to release spinlock
and set the interrupt status back to its former glory.

\vspace{1pc}

2. This is how we understood the conditional statements, but reading
at people's reaction to giving lock as an argument to condition signal
makes it seem this could be wrong, we hope to get a better explanation
for this part, the code is pretty self explanatory, as short as it is.


\section{One-Lane Synchronisation}

1. When syncronising a stretch of road like this, it would be
preferrable to account for different times of day when traffic amount
changes. But if we consider a simple solution:

Let the first car that gets to the crossing be put into the queue
first, let it pass. If no other cars get to the crossing while the
first is passing it, reset the queue when the car exits.

If more cars come, open up for new cars going in the same
direction. When a car approaches from the other direction, set a timer
of 1 minute before cars need to stop from the first direction, and
open up for the other direction, given the same rules. Again, if there
are no cars as the last crossing car leaves the narrow path, reset the
queue.

2. We did not manage to complete this part of the task yet.

3. And not this either.




%\pagenumbering{roman}
%\setcounter{tocdepth}{3}
%\thispagestyle{empty}
%\clearpage
%\renewcommand{\labelitemi}{$\cdot$}
%\begin{thebibliography}{9}
\end{thebibliography}


%\setlength{\parindent}{2em}
%\setlength{\parskip}{2ex}
%\renewcommand{\arraystretch}{1.25}
%\setcounter{page}{1}
\end{document}

